\documentclass{article}
\usepackage{listings}
\usepackage{xcolor}
\usepackage{hyperref}

\definecolor{codebackground}{rgb}{0.95,0.95,0.95}
\definecolor{codekeyword}{rgb}{0.13,0.29,0.53}
\definecolor{codestring}{rgb}{0.31,0.60,0.02}
\definecolor{codecomment}{rgb}{0.5,0.5,0.5}

\lstset{
  backgroundcolor=\color{codebackground},
  keywordstyle=\color{codekeyword}\bfseries,
  stringstyle=\color{codestring},
  commentstyle=\color{codecomment},
  breaklines=true,
  showstringspaces=false,
  language=Python
}

\title{MentorBI Project Documentation}
\author{Data Analysis Project}
\date{\today}

\begin{document}

\maketitle

\section{Project Overview}
This document summarizes a data analysis project using Python and SQL to analyze e-commerce data.

\section{Data Extraction}
The project extracts data from an SQL database using the following code:

\begin{lstlisting}
query = "SELECT * FROM ecommerce"
df = pd.read_sql(query, conn)
\end{lstlisting}

This code retrieves all records from the \texttt{ecommerce} table and loads them into a pandas DataFrame named \texttt{df} for further analysis.

\section{Technical Components}
\begin{itemize}
  \item \textbf{Database Connection}: The code uses a pre-established database connection object \texttt{conn}.
  \item \textbf{SQL Query}: A simple SELECT statement retrieves all data from the ecommerce table.
  \item \textbf{Data Framework}: Pandas is used to store and manipulate the retrieved data.
\end{itemize}

\section{Next Steps}
Potential next steps in this analysis could include:
\begin{itemize}
  \item Data cleaning and preprocessing
  \item Exploratory data analysis
  \item Visualization of key metrics
  \item Statistical analysis of e-commerce patterns
  \item Machine learning model development for predictions
\end{itemize}

\end{document}
