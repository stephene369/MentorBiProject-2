\documentclass{article}
\usepackage{graphicx}
\usepackage{amsmath}
\usepackage{hyperref}
\usepackage[T1]{fontenc}


\title{Analyse des Avis Clients et des Comportements de Consommation}
\author{Spero TESSY \\ Stephene WANTCHEKON}
\date{\today}

\begin{document}
	
	\maketitle
	
	\section{Méthodes d'Analyse Marketing}
	
	\subsection{Questions :}
	\begin{itemize}
		\item \textbf{Quel est l'impact des avis clients sur la performance des ventes ?}
		\begin{itemize}
			\item Utiliser des métriques de satisfaction client (par exemple, la proportion de produits recommandés) pour identifier les produits à forte performance.
			\item Calculer la moyenne des notes par produit et déterminer la relation avec les ventes.
		\end{itemize}
		
		\item \textbf{Comment optimiser les campagnes marketing pour améliorer les avis des clients ?}
		\begin{itemize}
			\item Quels types de produits ou catégories génèrent des avis positifs ?
			\item Quelle stratégie marketing pourrait être mise en place pour augmenter les avis positifs ?
		\end{itemize}
		
		\item \textbf{Quels produits génèrent le plus d'avis positifs ?}
		\begin{itemize}
			\item Identifier les produits les mieux notés et observer leur performance de vente.
			\item Comparer les produits ayant un grand nombre d'avis avec ceux ayant peu d'avis pour déterminer s'il existe une corrélation entre la quantité d'avis et la performance de vente.
		\end{itemize}
	\end{itemize}
	
	\subsection{Méthode :}
	\begin{itemize}
		\item Analyser les \textbf{produits les mieux notés} et observer leur \textbf{fréquence d'achat}.
		\item Utiliser des \textbf{modèles de régression simple} pour voir l'impact des caractéristiques d'un produit (prix, catégorie, etc.) sur les recommandations.
	\end{itemize}
	
	\subsection{Outils :}
	\begin{itemize}
		\item Utiliser \texttt{pandas} pour les calculs de moyennes et pour filtrer les données.
		\item Utiliser \texttt{seaborn} et \texttt{matplotlib} pour visualiser la distribution des avis et la relation avec les ventes.
	\end{itemize}
	
	\section{Compléter les Données avec le Prix}
	
	\subsection{Questions :}
	\begin{itemize}
		\item \textbf{Comment trouver les prix des articles pour compléter la base de données ?}
		\begin{itemize}
			\item Rechercher les prix des articles à partir des sources externes (site web de l’entreprise, par exemple, ou utiliser des bases de données publiques).
			\item Ajouter ces informations dans la base de données pour enrichir l’analyse.
		\end{itemize}
		
		\item \textbf{Analyse des relations entre le prix et les avis :}
		\begin{itemize}
			\item Une fois les prix ajoutés, analyser les relations entre \textbf{prix} et \textbf{avis clients}.
		\end{itemize}
	\end{itemize}
	
	\subsection{Explication :}
	\begin{itemize}
		\item Cela permet de mieux comprendre les effets du prix sur les comportements des clients et sur les avis.
		\item Par exemple, les produits plus chers génèrent-ils des avis plus positifs ou négatifs ?
	\end{itemize}
	
	\section{Analyse du Sentiment et des Commentaires des Clients}
	
	\subsection{Questions :}
	\begin{itemize}
		\item \textbf{Quel est le sentiment général des clients dans les commentaires ?}
		\begin{itemize}
			\item Comment les clients se sentent-ils vis-à-vis des produits ? Utiliser des techniques d'analyse de sentiment pour identifier les commentaires positifs ou négatifs.
		\end{itemize}
		
		\item \textbf{Les produits les mieux notés sont-ils associés à des sentiments positifs spécifiques ?}
		\begin{itemize}
			\item Les avis qui mentionnent des mots comme « satisfait », « qualité » ou « confort » ont-ils tendance à avoir des notes plus élevées ?
		\end{itemize}
	\end{itemize}
	
	\subsection{Méthodes :}
	\begin{itemize}
		\item \textbf{Analyse de sentiment :} Utiliser des bibliothèques Python comme \texttt{VADER}, \texttt{TextBlob}, ou \texttt{NLTK} pour déterminer le sentiment des avis clients. Ces outils peuvent classer les commentaires comme positifs, négatifs ou neutres.
		\item \textbf{Fréquence des mots et relations avec les notes :}
		\begin{itemize}
			\item Analyser la fréquence des mots associés à des commentaires positifs ou négatifs (ex. « mauvaise qualité », « très satisfait »).
		\end{itemize}
	\end{itemize}
	
	\subsection{Outils :}
	\begin{itemize}
		\item \texttt{VADER} : Utiliser \texttt{nltk.sentiment.vader.SentimentIntensityAnalyzer} pour évaluer les sentiments des avis.
		\item \texttt{TextBlob} : Utiliser \texttt{TextBlob} pour analyser le sentiment global des commentaires textuels.
	\end{itemize}
	
	\section{Analyse du Comportement du Consommateur}
	
	\subsection{Questions :}
	\begin{itemize}
		\item \textbf{Quels sont les facteurs influençant le comportement d'achat ?}
		\begin{itemize}
			\item L’âge des clients, les catégories de produits et les prix influencent-ils les décisions d'achat ?
		\end{itemize}
		
		\item \textbf{Quel type de comportement montre les clients lors de l'achat des produits les mieux notés ?}
		\begin{itemize}
			\item Les clients qui achètent des produits chers ont-ils une probabilité plus élevée de laisser des commentaires ?
		\end{itemize}
	\end{itemize}
	
	\subsection{Méthode :}
	\begin{itemize}
		\item \textbf{Segmentation des clients} en fonction de l'âge et des catégories de produits qu'ils achètent.
		\item Utiliser des \textbf{groupes d’âge} et analyser la distribution des notes pour chaque segment.
		\item \textbf{Relation entre le nombre d'achats et la notation} :
		\begin{itemize}
			\item Visualiser si les clients ayant effectué plusieurs achats dans la même catégorie de produit donnent des avis plus détaillés ou plus positifs.
		\end{itemize}
	\end{itemize}
	
	\subsection{Outils :}
	\begin{itemize}
		\item \texttt{Pandas} pour segmenter les données et effectuer des analyses descriptives par groupes.
		\item \texttt{Matplotlib/Seaborn} pour la visualisation des relations entre l’âge, la catégorie de produit, et les avis.
	\end{itemize}
	
	\section{Sentiment des Clients et Impact sur les Recommandations}
	
	\subsection{Questions :}
	\begin{itemize}
		\item \textbf{Est-ce que le sentiment d'un client affecte sa propension à recommander un produit ?}
		\begin{itemize}
			\item Les clients ayant un sentiment positif recommandent-ils plus souvent un produit que ceux ayant un sentiment négatif ?
		\end{itemize}
	\end{itemize}
	
	\subsection{Méthode :}
	\begin{itemize}
		\item \textbf{Cross-tabulation} : Comparer le sentiment des commentaires avec la colonne "Recommended IND" pour voir si un sentiment positif est plus fortement lié à une recommandation.
	\end{itemize}
	
	\subsection{Outils :}
	\begin{itemize}
		\item \texttt{Pandas} pour la création de matrices croisées.
		\item \texttt{VADER/TextBlob} pour l'analyse du sentiment.
	\end{itemize}
	
	\section{Impact de l'Âge et du Prix sur les Avis des Clients}
	
	\subsection{Questions :}
	\begin{itemize}
		\item \textbf{Les avis sont-ils influencés par l'âge du client et par le prix du produit ?}
		\begin{itemize}
			\item Existe-t-il des différences dans la manière dont les jeunes et les adultes évaluent les produits en fonction du prix ?
		\end{itemize}
	\end{itemize}
	
	\subsection{Méthode :}
	\begin{itemize}
		\item \textbf{Corrélation entre l'âge et le prix} : Analyser si les jeunes sont plus susceptibles d'acheter des produits bon marché et s’ils donnent des avis plus positifs ou négatifs en conséquence.
	\end{itemize}
	
	\subsection{Outils :}
	\begin{itemize}
		\item \texttt{Pandas} pour effectuer la corrélation et l'agrégation des données.
		\item \texttt{Seaborn/Matplotlib} pour visualiser les relations entre l’âge, le prix, et les notes.
	\end{itemize}
	
\end{document}
